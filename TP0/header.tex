\usepackage[margin=2.5cm,a4paper]{geometry}
\usepackage{framed,fancyhdr,xcolor,fvextra,newunicodechar, float}

% Declara simbolos de terminal zsh para poder copypastear desde la terminal
\newunicodechar{➜}{\ensuremath{\rightarrow}}
\newunicodechar{✗}{\ensuremath{\triangle}}
\newunicodechar{✓}{\ensuremath{\triangle}}

% Sacar 'Figure 1:' de las imagenes
\usepackage[labelformat=empty]{caption}

\newcommand{\nombre}{Federico del Mazo}
\newcommand{\materia}{[75.42] Taller de Programación}
\newcommand{\cuatri}{2019c2}
\newcommand{\trabajo}{TP0: Contador de Palabras}

\pagestyle{fancy}
\fancyhead[L]{\materia  \\ \trabajo}
\fancyhead[R]{\nombre \\ \cuatri}

% Hace letra monoespaciada en magenta
\let\OldTexttt\texttt
\renewcommand{\texttt}[1]{\OldTexttt{\color{magenta}{#1}}}    

% Pone barra magenta a la izquierda de los bloques de codigo, sean con o sin lenguaje especificado
% Shaded -> Lenguaje especificado
% Verbatim -> Lenguaje no especificado
\newenvironment{leftbar_mod}{
	\def\FrameCommand{{\color{magenta}\vrule width 1pt} \hspace{10pt}}
	\MakeFramed {\advance\hsize-\width \FrameRestore}}
{\endMakeFramed}
\let\oldshaded\Shaded
\renewenvironment{Shaded}{\begin{leftbar_mod}\begin{oldshaded}}{\end{oldshaded}\end{leftbar_mod}}
\let\oldverbatim\verbatim
\renewenvironment{verbatim}{\begin{leftbar_mod}\begin{oldverbatim}}{\end{oldverbatim}\end{leftbar_mod}}

% Code wrap cuando hay lenguaje especificado
\DefineVerbatimEnvironment{Highlighting}{Verbatim}{breaklines,commandchars=\\\{\}}

% Siempre que se puede, posicionar las imagenes en el lugar que se las indica
\let\origfigure\figure
\let\endorigfigure\endfigure
\renewenvironment{figure}[1][2] {
    \expandafter\origfigure\expandafter[H]
} {
    \endorigfigure
}